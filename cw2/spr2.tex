\documentclass[fleqn,10pt,a4paper]{article}
\usepackage{polski}
\usepackage[T1]{fontenc}
\usepackage[utf8]{inputenc}
\usepackage[a4paper, left=2.5cm, right=2.5cm, top=2.5cm, bottom=2.5cm]{geometry} 
\usepackage{enumerate}
\usepackage{sidecap}
\usepackage{wrapfig}
\usepackage{subfig}
\usepackage{fancyhdr}
\usepackage{multirow}
\usepackage{gensymb}
\usepackage{graphicx}
\usepackage{url}
\usepackage{xurl}
\usepackage{hyperref}
\usepackage{array}
\usepackage{amsmath}
\usepackage{textcomp}
\usepackage{verbatim}
\usepackage{media9}
\usepackage{listings}
\usepackage{color}
\usepackage{listings}
\usepackage{color}

\definecolor{mygreen}{rgb}{0,0.6,0}
\definecolor{mygray}{rgb}{0.5,0.5,0.5}
\definecolor{mymauve}{rgb}{0.58,0,0.82}

\lstset{
  language=Python,
  basicstyle=\footnotesize\ttfamily,
  commentstyle=\color{mygreen},
  keywordstyle=\color{blue},
  numberstyle=\tiny\color{mygray},
  numbers=none,
  stringstyle=\color{mymauve},
  breaklines=true,
  showstringspaces=false,
  frame=single,
  backgroundcolor=\color{white},
  literate={ą}{{\k{a}}}1 {ć}{{\'c}}1 {ę}{{\k{e}}}1 {ł}{{\l}}1 {ń}{{\'n}}1 {ó}{{\'o}}1 {ś}{{\'s}}1 {ź}{{\'z}}1 {ż}{{\.z}}1
}

\hypersetup{
  colorlinks=true,
  linkcolor=black,
  filecolor=magenta,      
  urlcolor=cyan,
}

\hypersetup{breaklinks=true}
\urlstyle{same}
\renewcommand{\lstlistingname}{Kod źródłowy}
\renewcommand{\lstlistlistingname}{Spis kodów źródłowych}
\renewcommand{\figurename}{Wykres}
\renewcommand{\listfigurename}{Spis wykresów}

\makeatletter
\renewcommand{\maketitle}{%
  \begin{titlepage}
    \begin{center}
      \vspace*{2cm}
      {\huge \@title \par}
      \vspace{1.5cm}
      {\large Dane nr 15\par}
      \vspace{1.5cm}
      {\large \@author \\ 325693 \par}
      {\large Wydział Geodezji i Kartografii\\ Politechnika Warszawska \par}
      \vspace{9cm}
      {\large \@date \par}
      \vspace{1.5cm}
    \end{center}
  \end{titlepage}
}
\makeatother
    \title{\textbf{SPRAWOZDANIE Z ĆWICZENIA 2:}\\ Transformacja współrzędnych elipsoidalnych samolotu do układu lokalnego}
\author{Maja Kret}
\date{Warszawa, \today}


\setlength{\parindent}{0cm}
\setlength{\parskip}{1ex plus 0.5ex minus 0.2ex}
\linespread{1.3}


\begin{document} 
\pagestyle{fancy}
\fancyhf{}
\rfoot{\thepage}
\renewcommand{\headrulewidth}{0pt}
\maketitle
\rhead{~}

\tableofcontents
\newpage

\section{Cel ćwiczenia}
Celem ćwiczenia jest transformacja współrzędnych elipsoidalnych lecącego samolotu do układu lokalnego. 
Następnie należy zwizualizować trasę lotu na mapach i wykresach współrzędnych.

\section{Wstęp teoretyczny}

Współrzędne elipsoidalne są powszechnie stosowane w geodezji, reprezentując położenie punktu 
na powierzchni elipsoidy odniesienia za pomocą szerokości ($\varphi$) i długości ($\lambda$) geograficznej oraz wysokości ($h$) nad powierzchnią 
elipsoidy. W kontekście lotnictwa, te współrzędne pozwalają na precyzyjne określenie lokalizacji samolotu w danym momencie lotu. 
Przekształcenie tych danych do układu lokalnego, związanego z konkretnym punktem na Ziemi - w tym przypadku lotniskiem - umożliwia lepsze 
zrozumienie względnej pozycji samolotu w odniesieniu do lokalizacji lotniska, a co za tym idzie, stwierdzenie, czy samolot znajduje się w zasięgu widzenia.
W przypadku, gdy kąt elewacji jest ujemny, samolot znajduje się poniżej horyzontu i nie jest widoczny z lotniska.

\section{Dane do ćwiczenia}
\subsection{Dane lotu}
Danymi do ćwiczenia jest plik \texttt{lot15.csv} zawierający dane o locie.
Analizowany był lot o numerze LOT4YH - LO483 z Warszawy do Oslo.
Każdy wiersz pliku zawiera następujące dane:
\begin{table}[!ht]
\centering
\begin{tabular}{|c|c|c|}
\hline
Indeks & Nazwa & Jednostka \\ \hline
0 & \multicolumn{2}{c|}{Znacznik czasu} \\ \hline
1 & \multicolumn{2}{c|}{rok}  \\ \hline
2 & \multicolumn{2}{c|}{miesiąc} \\ \hline
3 & \multicolumn{2}{c|}{dzień} \\ \hline
4 & \multicolumn{2}{c|}{godzina} \\ \hline
5 & \multicolumn{2}{c|}{minuta} \\ \hline
6 & \multicolumn{2}{c|}{sekunda} \\ \hline
7 & Szerokość geograficzna $[\varphi]$ & \degree \\ \hline
8 & Długość geograficzna $[\lambda]$ & \degree \\ \hline
9 & Wysokość $[h]$ & stopy \\ \hline
10 & Prędkość & kts \\ \hline
11 & Kierunek & $\degree$ \\ \hline
\end{tabular}
\caption{Dane samolotu \label{dane_excel}}
\end{table}
\newpage
\subsection{Dane elipsoidy}
\begin{lstlisting}[language=Python, caption = Dane elipsoidy,label = dane_lotniska]
  h_norm_waw = 104
  h_norm_osl = 208
  undulacja = 31.4
  a = 6378137.0
  e2 = 0.00669438002290
\end{lstlisting}
gdzie $h\_norm$ to wysokość normalna dla danego lotniska, a $a$ i $e2$ to parametry elipsoidy GRS80


\section{Przebieg ćwiczenia}

\begin{enumerate}
  \item \textbf{Wczytanie danych z pliku:} Dane dotyczące badanego lotu zostały wczytane z pliku \texttt{lot15.csv} za pomocą funkcji \texttt{read\_flightradar}.
  \item \textbf{Selekcja danych lotu:} Wydobyto dane, zebrane podczas, gdy samolot znajdował się ponad poziomem Ziemi - miał dodatnią wysokość.
  Interesują nas współrzędne samolotu, jego prędkość i wysokość oraz dokładna godzina, o której te parametry zostały zarejestrowane.  
  Następnie wyselekcjonowano współrzędne lotnisk odlotu i przylotu. Do wysokości dodano też wysokość normalną dla danego lotniska oraz undulację.
  Undulację dla Oslo pominięto. Prędkość przeliczono z węzłów na $\frac{km}{h}$. A wysokość samolotu z stóp na metry.
  \item \textbf{Przeliczanie współrzędnych lotniska i samolotu do współrzędnych ortokartezjańskich:} 
  Funkcja \texttt{orto} z kodu źródłowego \ref{transformacja} przyjmuje jako argumenty szerokość $[\varphi]$ i długość $[\lambda]$ geograficzną w radianach oraz wysokość $[h]$
  w metrach. Następnie przelicza i zwraca współrzędne ortokartezjańskie $[X, Y, Z]$.
  \item \textbf{Transformacja wektora współrzędnych samolotu do współrzędnych lokalnych:} 
  Transformacja odbywa się za pomocą macierzy obrotu $R$ tworzonej w funkcji \texttt{macierz\_obrotu} przyjmującej jako argumenty 
  szerokość $[\varphi]$ i długość $[\lambda]$ geograficzną w radianach. Funkcja znajduje się w kodzie źródłowym \ref{transformacja}.
  \item \textbf{Obliczenia azymutu i elewacji:} W pętli opisanej w kodzie źródłowym \ref{petla}, dla każdej z pozycji samolotu obliczono azymut i elewację względem lotnisk
  oraz odległość od lotniska. Dane zapisano w tablicach \texttt{data\_waw} i \texttt{data\_osl} wraz z czasem.
  \item \textbf{Wizualizacje:} Do przygotowania wykresów wykorzystano bibliotekę \texttt{matplotlib}, a do map lotu - \texttt{plotly} oraz \texttt{cartopy}.
  Wykonano mapy trasy samolotu \ref{linia} i \ref{newplot} oraz wykresy \ref{skyplot_waw} - \ref{skyplot_odl_osl} pokazujące znikanie i pojawianie się samolotu nad horyzontem.
  Następnie opracowano wykresy \ref{wys_h} - \ref{odleglosc_h}: wysokości, prędkości, kąta elewacji i odległości samolotu od lotniska w funkcji czasu.
  \item \textbf{Opracowanie wyników} Na podstawie obliczonych wartości stwierdzono w jakich odległościach od lotniska samolot był widoczny nad horyzontem,
  a także w jakich godzinach znikał i się pojawiał. Zebrano także informacje o maksymalnej wysokości, prędkości.
\end{enumerate}

\newpage
\section{Wyniki ćwiczenia}
\subsection{Mapa lotu}


\begin{figure}[ht!]
  \centering
  \includegraphics[width=1\textwidth]{zdjecia/linia_geo.png}
  \caption{Linia lotu}
  \label{linia}
\end{figure}
Mapa \ref{linia} przedstawia linię geodezyjną pomiędzy lotniskami przylotu i odlotu będąca odwzorowaniem idealnego przebiegu trasy.
Trasa lotu ma długość 1076.99 km, a czas przelotu wynosi 1h 45min.

\begin{figure}[ht!]
  \centering
  \includegraphics[width=0.45\textwidth]{zdjecia/newplot.png}
  \includegraphics[width=0.45\textwidth]{zdjecia/newplot_europe.png}
  \caption{Mapa lotu}
  \label{newplot}
\end{figure}
Mapa nr \ref{newplot} przedstawia trasę samolotu stworzoną z kolejnych zarejestrowanych lokalizacji samolotu.
  
\newpage

\subsection{Wykresy Skyplot dla lotniska WAW}
\begin{figure}[h!]
  \centering
  \includegraphics[width=1\textwidth]{zdjecia/odl_waw_punkty.png}
  \caption{Wykres odległości samolotu od azymutu z lotniska WAW}
  \label{skyplot_waw}
\end{figure}

Wykres \ref{skyplot_waw} przedstawia kolejne odległości samolotu w funkcji azymutu aż do momentu zniknięcia poniżej
horyzontu o godzinie 15:48 w odległości 321.52 km od lotniska. W tym momencie kąt elewacji samolotu schodzi poniżej $0 \degree$.

\begin{figure}[ht!]
  \centering
  \includegraphics[width=1\textwidth]{zdjecia/odl_waw_horyzont.png}
  \caption{Wykres odległości samolotu od azymutu z lotniska WAW w postaci linii}
  \label{skyplot_odl_waw}
\end{figure}

Na wykresie \ref{skyplot_odl_waw} również zaznaczona jest zależność odległości samolotu od azymutu z lotniska Warszawie.
Zawarte są tu dane z całego lotu, a interesują nas są te zaznaczone na czerwono - od momentu startu do zniknięcia samolotu za horyzontem.

\clearpage

\subsection{Wykresy Skyplot dla lotniska OSL}
\begin{figure}[h!]
  \centering
  \includegraphics[width=1\textwidth]{zdjecia/odl_osl_punkty.png}
  \caption{Wykres odległości samolotu od azymutu z lotniska OSL}
  \label{skyplot_osl}
\end{figure}

Wykres \ref{skyplot_osl} pokazuje odległości samolotu od lotniska OSL w funkcji azymutu, w momencie gdy samolot znajduje się powyżej horyzontu.
Pojawienie się samolotu następuje o godzinie 16:24 w odległości 365.43 km od lotniska. Jest to pierwsza zarejestrowana odległość od lotniska, gdzie wartość
kąta elewacji jest dodatnia.

\begin{figure}[ht!]
  \centering
  \includegraphics[width=1\textwidth]{zdjecia/odl_osl_horyzont.png}
  \caption{Wykres Skyplot odległości samolotu od azymutu z lotniska OSL w postaci linii}
  \label{skyplot_odl_osl}
\end{figure}

Powyższy wykres \ref{skyplot_odl_osl} przedstawia te same dane w postaci ciągłej linii reprezentującej całą trasę samolotu.
Wyselekcjonowano tu na czerwono momenty, gdy samolot był widoczny z Oslo.

\clearpage

\subsection{Wykresy w funkcji czasu}

\begin{figure}[ht!]
  \centering
  \includegraphics[width=1\textwidth]{zdjecia/wysokosc_h.png}
  \caption{Wykres zależności wysokości samolotu od czasu}
  \label{wys_h}
\end{figure}

Na wykresie \ref{wys_h} zobrazowano wysokości samolotu od chwili startu o 15:19 do lądowania o 17:04. 
Nie zawarto tu danych o samolocie podczas gdy przebywał na ziemi i wysokość wynosiła 0 m. Po osiągnięciu maksymalnej wysokości
10.370 km, samolot pozostaje na tym poziomie większość lotu, aż do momentu hamowania przed lądowaniem.

\begin{figure}[ht!]
    \centering
    \includegraphics[width=1\textwidth]{zdjecia/predkosc_h.png}
    \caption{Wykres zależności prędkości samolotu od czasu}
    \label{pred_h}
  \end{figure}

Wykres \ref{pred_h} zawiera zarejestrowane prędkości samolotu. Podczas wznoszenia się samolotu, w przeciągu 15 min, prędkość samolotu rośnie z $300 \frac{km}{h}$ - 
podczas startu aż do $750 \frac{km}{h}$ -  w momencie osiągnięcia wysokości ponad 10 km. Następnie prędkość spada do $700 \frac{km}{h}$ i utrzymuje się na wysokim 
poziomie do 20 minut przed lądowaniem. Wraz z zmniejszeniem prędkości maleje też wysokość samolotu.
Największa zarejestrowana prędkość wynosi aż $840.808 \frac{km}{h}$, natomiast średnia z całego lotu to 615.135 $\frac{km}{h}$.

\begin{figure}[ht!]
  \centering
  \includegraphics[width=1\textwidth]{zdjecia/elewacja.png}
  \caption{Wykres zależności wysokości kątowej samolotu od czasu z lotniska WAW}
  \label{elewacja_h}
\end{figure}

Kąt elewacji, którego wartości przedstawiono na wykresie \ref{elewacja_h} maleje wraz z czasem. Można doczytać się przybliżonej
godziny zniknięcia samolotu poniżej horyzontu gdy wykres osiąga wartość $0 \degree$.

\begin{figure}[ht!]
  \centering
  \includegraphics[width=1\textwidth]{zdjecia/odlod_lotniska.png}
  \caption{Wykres zależności odległości samolotu od lotniska WAW od czasu}
  \label{odleglosc_h}
\end{figure}

Wykres \ref{odleglosc_h} przedstawia zmianę odległość samolotu od lotniska odlotu. Jest ona stale rosnąca w czasie.

\clearpage
\section{Kod programu}

\begin{lstlisting}[language=Python, caption=Funkcja wczytywania danych udostępniona przez prowadzącego, label = read]
  def read_flightradar(file):
      with open(file, 'r') as f:
          i = 0
          size= []
          Timestamp = []; date = []; UTC = []; Latitude = []; Longitude = []; 
          Altitude = []; Speed = []; Direction = []; datetime_date = []
          for linia in f:
              if linia[0:1]!='T':
                  splited_line = linia.split(',')
                  size.append(len(splited_line))
                  i+=1
                  Timestamp.append(int(splited_line[0]))
                  full_date = splited_line[1].split('T')
                  date.append(list(map(int,full_date[0].split('-'))))
                  UTC.append(list(map(int, full_date[1].split('Z')[0].split(':'))))
                  Callsign = splited_line[2]
                  Latitude.append(float(splited_line[3].split('"')[1]))
                  Longitude.append(float(splited_line[4].split('"')[0]))
                  Altitude.append(float(splited_line[5]))
                  Speed.append(float(splited_line[6]))
                  Direction.append(float(splited_line[7]))
                  
      all_data = np.column_stack((np.array(Timestamp), np.array(date), np.array(UTC),
                                  np.array(Latitude), np.array(Longitude), np.array(Altitude),
                                  np.array(Speed), np.array(Direction)))
      return all_data
\end{lstlisting}

\begin{lstlisting}[language=Python, caption=Funkcje transformacji współrzędnych, label = transformacja]
  # Funkcja przeliczająca współrzędne geograficzne na współrzędne ortogonalne
  def orto(p, l, h):
      N = a / (np.sqrt(1 - e2 * np.sin(l) * np.sin(l)))
      X = (N + h) * np.cos(p) * np.cos(l)
      Y = (N + h) * np.cos(p) * np.sin(l)
      Z = (N * (1 - e2) + h) * np.sin(p)
      return([X, Y, Z])

  # Funkcja obliczająca macierz obrotu
  def macierz_obrotu(p, l):
      macierz = np.array([[-np.sin(p) * np.cos(l), -np.sin(l), np.cos(p) * np.cos(l)],
                          [-np.sin(p) * np.sin(l), np.cos(l), np.cos(p) * np.sin(l)],
                          [np.cos(p), 0, np.sin(p)]])
      return macierz

\end{lstlisting}
\newpage
\begin{lstlisting}[language=Python, caption=Selekcja danych, label = dane]
  # Współrzędne
  wspolrzedne = dane[:, 7:10]
  lot = np.where(wspolrzedne[:, -1]>0)[0]
  wspolrzedne[:, -1] = wspolrzedne[:, -1] * 0.3048
  wspolrzedne_lot = wspolrzedne[lot, :]
  wspol_waw = wspolrzedne[lot[0]-1, :]
  wspol_osl = wspolrzedne[lot[-1]+1, :]

  # Czas
  time_hms = dane[:, 4:7]
  time_hms_lot = time_hms[lot, :]
  time = time_hms[:, 0] * 60 + time_hms[:, 1] + time_hms[:, 2] / 60
  time_lot = time[lot]
  time_lot = time_lot - time_lot[0]

  # Start i lądowanie
  take_off = datetime.datetime(int(dane[lot[0], 1]), int(dane[lot[0], 2]), int(dane[lot[0], 3]), int(dane[lot[0], 4]), int(dane[lot[0], 5]), int(dane[lot[0], 6]))
  landing = datetime.datetime(int(dane[lot[-1], 1]), int(dane[lot[-1], 2]), int(dane[lot[-1], 3]), int(dane[lot[-1], 4]), int(dane[lot[-1], 5]), int(dane[lot[-1], 6]))
  flight_time = landing - take_off

  # Oś czasu dla lotu
  datetime_time_lot = [(take_off + datetime.timedelta(minutes = i)).strftime('%H:%M') for i in time_lot]
  datetime_time_lot = np.array(datetime_time_lot)
  datetime_time_lot = pd.to_datetime(datetime_time_lot)

  # Prędkość
  speed = dane[:, 10] * 1.852
  speed_lot = speed[lot]

  # Współrzędne lotniska WAW
  phi_waw = np.deg2rad(wspol_waw[0])
  lambda_waw = np.deg2rad(wspol_waw[1])
  h_waw = wspol_waw[2] + h_norm_waw + undulacja

  xyz_waw = orto(phi_waw, lambda_waw, h_waw)
  R_waw = macierz_obrotu(phi_waw, lambda_waw)

  # Wspolrzedne lotniska OSL
  phi_osl = np.deg2rad(wspol_osl[0])
  lambda_osl = np.deg2rad(wspol_osl[1])
  h_osl = wspol_osl[2] + h_norm_osl

  xyz_osl = orto(phi_osl, lambda_osl, h_osl)
  R_osl = macierz_obrotu(phi_osl, lambda_osl)
\end{lstlisting}
\newpage
\begin{lstlisting}[language = Python, caption = Transformacja współrzędnych, label = petla]
  # Azymut
  azimuths = []
  azimuths_o = []
  azimuths_waw = []
  azimuths_osl = []
  # Odległość od lotniska
  distance = []
  distance_o = []
  distance_waw = []
  distance_osl = []
  # Wysokość
  height = []
  height_waw = []
  height_osl = []
  # Czas
  time_waw = []
  time_osl = []
  
  # Warszawa 
  for flh in wspolrzedne_lot:
      xyz_samolotu = orto(np.deg2rad(flh[0]), np.deg2rad(flh[1]), flh[2])
      wektor_samolot_lotnisko = np.array([xyz_samolotu[0] - xyz_waw[0], xyz_samolotu[1] - xyz_waw[1], xyz_samolotu[2] - xyz_waw[2]])
      neu = R_waw.T@wektor_samolot_lotnisko
      h = np.arcsin(np.clip(neu[2] / np.sqrt(neu[0]**2 + neu[1]**2 + neu[2]**2), -1, 1))
      h = np.degrees(h)
      height.append(h)
  
      az = np.arctan2(neu[1], neu[0])
      azimuths.append(az)
  
      dist = np.sqrt(neu[0]**2 + neu[1]**2 + neu[2]**2) / 1000
      distance.append(dist)
  
      if h > 0:
          height_waw.append(h)
          distance_waw.append(dist)
          azimuths_waw.append(az)
          time_waw.append(flh[2])
  
  # Oslo
  for flh in wspolrzedne_lot:
      xyz_samolotu = orto(np.deg2rad(flh[0]), np.deg2rad(flh[1]), flh[2])
      wektor_samolot_lotnisko = np.array([xyz_samolotu[0] - xyz_osl[0], xyz_samolotu[1] - xyz_osl[1], xyz_samolotu[2] - xyz_osl[2]])
      neu = R_osl.T @ wektor_samolot_lotnisko
  
      h = np.arcsin(np.clip(neu[2] / np.sqrt(neu[0]**2 + neu[1]**2 + neu[2]**2), -1, 1))
      h = np.degrees(h)
  
      az = np.arctan2(neu[1], neu[0])
      az = az if az >= 0 else az + 360
      azimuths_o.append(az)
      
      dist = np.sqrt(neu[0]**2 + neu[1]**2 + neu[2]**2) / 1000
      distance_o.append(dist)
  
      if h > 0:
          height_osl.append(h)
          distance_osl.append(dist)
          azimuths_osl.append(az)
          time_osl.append(flh[2])
  
  # Tablica współrzędnych i czasu
  data_waw = np.column_stack((azimuths_waw, np.array(height_waw), np.array(distance_waw), np.array(time_waw)))
  data_osl = np.column_stack((azimuths_osl, np.array(height_osl), np.array(distance_osl), np.array(time_osl)))
  
\end{lstlisting}
\newpage
\begin{lstlisting}[language = Python, caption = Mapy lotu, label = mapy]
  # Linia geodezyjna lotu
  fig = plt.figure(figsize=(10, 5))
  ax = plt.axes(projection=ccrs.Robinson())
  request = cimgt.OSM()
  ax.add_image(request, 6)
  ax.stock_img()
  ax.coastlines()
  # ax.plot(wspolrzedne_lot[:,1], wspolrzedne_lot[:,0],transform=ccrs.PlateCarree(),color='b')
  ax.plot([wspol_waw[1], wspol_osl[1]], [wspol_waw[0], wspol_osl[0]], transform=ccrs.Geodetic(), color='r')
  extent = [-5, 35, 50, 65]
  ax.set_extent(extent)

  # Mapa lokalizacji samolotu
  fig = px.scatter_geo(wspolrzedne_lot,
          lat=wspolrzedne_lot[:, 0], 
          lon=wspolrzedne_lot[:, 1], 
          color = time_lot, size = 1, 
          title = 'Zarejestrowane współrzędne samolotu', 
          hover_name = time_lot, 
          projection="orthographic")
  fig.update_geos(fitbounds="locations")
  fig.update_layout(width = 1000, height = 1000)
  fig.show()

  # Trasa lotu
  fig = go.Figure(data=go.Scattergeo(
          lon = wspolrzedne_lot[:, 1],
          lat = wspolrzedne_lot[:, 0],
          mode = 'lines',
          line = dict(width = 2, color = 'blue')))
  fig.update_geos(fitbounds="locations")
  fig.update_layout(title = 'LOT4YH', width = 2000, height = 1000)

  # Etykiety lotnisk
  fig.add_trace(go.Scattergeo(
          lon = [wspolrzedne_lot[0, 1], wspolrzedne_lot[-1, 1]],
          lat = [wspolrzedne_lot[0, 0], wspolrzedne_lot[-1, 0]],
          mode = 'markers + text',
          marker = dict(size = 10, color = 'navy'),
          text = ['WAW'],
          textposition="top center"))
  fig.add_trace(go.Scattergeo(
          lon = [wspolrzedne_lot[-1, 1], wspolrzedne_lot[-1, 1]],
          lat = [wspolrzedne_lot[-1, 0], wspolrzedne_lot[-1, 0]],
          mode = 'markers + text',
          marker = dict(size = 10, color = 'navy'),
          text = ['OSL'],
          textposition="top center"))
  fig.show()

\end{lstlisting}
\newpage
\begin{lstlisting}[language = Python, caption = Wykresy Skyplot pokazujące znikanie i pojawianie się samolotu nad horyzontem, label = skyplot]
  for data in (data_waw, data_osl):
    # Skyplot z elewacją punkty 
    fig, ax = plt.subplots()
    ax = plt.axes(projection = 'polar')
    ax.set_theta_zero_location('N')
    ax.set_theta_direction(-1)
    ax.set_xlabel('Azymut [deg]')
    ax.set_ylabel('Wysokość [deg]')
    sc = ax.scatter(data[:,0], data[:,1], c = data[:,3], cmap = 'viridis')
    ax.set_title('Elewacja od lotniska od azymutu')

    # Skyplot z elewacją w postaci linii geodezyjnej 
    fig, ax = plt.subplots()
    ax = plt.axes(projection = 'polar')
    ax.set_theta_zero_location('N')
    ax.set_theta_direction(-1)
    ax.set_xlabel('Azymut [deg]')
    ax.set_ylabel('Wysokość [deg]')
    ax.plot(data[:,0], data[:,1], color = 'blue')
    ax.set_title('Elewacja od lotniska od azymutu')

    # Skyplot z odległością punkty 
    fig, ax = plt.subplots()
    ax = plt.axes(projection = 'polar')
    ax.set_theta_zero_location('N')
    ax.set_theta_direction(-1)
    ax.set_xlabel('Azymut [deg]')
    ax.set_ylabel('Odległość [km]')
    #sc = ax.scatter(azimuths, distance, color = 'blue') 
    sc = ax.scatter(data[:,0], data[:,2], c = data[:,3], cmap = 'viridis')
    ax.set_title('Odległość od lotniska od azymutu')

  # Skyplot z odległością w postaci linii geodezyjnej - WAW
  fig, ax = plt.subplots()
  ax = plt.axes(projection = 'polar')
  ax.set_theta_zero_location('N')
  ax.set_theta_direction(-1)
  ax.grid(True)
  ax.plot(azimuths, distance, color = 'blue')
  ax.plot(data_waw[:,0], data_waw[:,2], color = 'red')
  ax.set_xlabel('Azymut [deg]')
  ax.set_ylabel('Odległość [km]')
  ax.set_title('Odległość od lotniska WAW od azymutu')

  # Skyplot z odległością w postaci linii geodezyjnej - OSL
  fig, ax = plt.subplots()
  ax = plt.axes(projection = 'polar')
  ax.set_theta_zero_location('N')
  ax.set_theta_direction(-1)
  ax.grid(True)
  ax.plot(azimuths_o, distance_o, color = 'blue')
  ax.plot(data_osl[:,0], data_osl[:,2], color = 'red')
  ax.set_xlabel('Azymut [deg]')
  ax.set_ylabel('Odległość [km]')
  ax.set_title('Odległość od lotniska OSL od azymutu')

  plt.show()

  print('Odległość w momencie zniknięcia: ', data_waw[-1, 2], 'km')
  print('Odległość w momencie pojawienia się: ', data_osl[1, 2], 'km')

  print('Czas lotu: ', flight_time)
  print('Długość trasy: ', distance[-1], 'km')
  print('Wysokość maksymalna: ', np.max(wspolrzedne_lot[:, 2])/1000, 'km')
  
  print('Prędkość średnia: ', distance[-1] / (flight_time.total_seconds()/3600), 'km/h')
  print('Prędkość maksymalna: ', np.max(speed_lot), 'km/h')
\end{lstlisting}

\begin{lstlisting}[language=Python, caption=Wykresy położenia samolotu od czasu, label = wykresy]
  # Wysokośc od czasu
  fig, ax = plt.subplots()
  ax.plot(datetime_time_lot, wspolrzedne_lot[:, 2]/1000, color = 'red')
  ax.xaxis.set_major_formatter(mdates.DateFormatter('%H:%M'))
  ax.grid(True)
  ax.set_xlabel('Godzina [hh:mm]')
  ax.set_ylabel('Wysokość [km]')
  ax.set_title('Wysokość samolotu w funkcji czasu')
  
  # Prędkość lotu od czasu
  fig, ax = plt.subplots()
  ax.plot(datetime_time_lot, speed_lot, color = 'orange')
  ax.xaxis.set_major_formatter(mdates.DateFormatter('%H:%M'))
  ax.grid(True)
  ax.set_xlabel('Godzina [h:min]')
  ax.set_ylabel('Prędkość [km/h]')
  ax.set_title('Prędkość samolotu w funkcji czasu')

  # Elewacja od godziny
  fig, ax = plt.subplots()
  ax.plot(datetime_time_lot, height, color = 'green')
  ax.xaxis.set_major_formatter(mdates.DateFormatter('%H:%M'))
  ax.grid(True)
  ax.set_xlabel('Godzina [h:min]')
  ax.set_ylabel('Wysokość [deg]')
  ax.set_title('Wysokość kątowa samolotu w funkcji czasu')
  
  # Odległość od godziny
  fig, ax = plt.subplots()
  ax.plot(datetime_time_lot, distance, color = 'blue')
  ax.xaxis.set_major_formatter(mdates.DateFormatter('%H:%M'))
  ax.grid(True)
  ax.set_xlabel('Godzina [h:min]')
  ax.set_ylabel('Odelgłość [km]')
  ax.set_title('Odległość samolotu od lotniska WAW w funkcji czasu')
\end{lstlisting}

\newpage
\listoftables
\listoffigures
\lstlistoflistings

\end{document}