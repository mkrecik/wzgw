\documentclass[fleqn,10pt,a4paper]{article}
\usepackage{polski}
\usepackage[T1]{fontenc}
\usepackage[utf8]{inputenc}
\usepackage[a4paper, left=2.5cm, right=2.5cm, top=2.5cm, bottom=2.5cm]{geometry} 
\usepackage{enumerate}
\usepackage{sidecap}
\usepackage{wrapfig}
\usepackage{subfig}
\usepackage{fancyhdr}
\usepackage{multirow}
\usepackage{gensymb}
\usepackage{graphicx}
\usepackage{url}
\usepackage{xurl}
\usepackage{hyperref}
\usepackage{array}
\usepackage{amsmath}
\usepackage{textcomp}
\usepackage{verbatim}
\usepackage{media9}
\usepackage{listings}
\usepackage{color}
\usepackage{listings}
\usepackage{color}
\usepackage{float}
\usepackage{xcolor}

% Define custom colors
\definecolor{mygreen}{RGB}{127,148,0}
\definecolor{mygray}{RGB}{80,80,80}
\definecolor{mymauve}{RGB}{251,98,246}
\definecolor{myblue}{RGB}{0, 166, 237}
\definecolor{myorange}{RGB}{246, 81, 29}

% Define a custom style for your code
\lstdefinestyle{mycode}{
  backgroundcolor=\color{white},
  commentstyle=\color{mygreen},
  keywordstyle=\color{myblue},
  numberstyle=\tiny\color{black},
  stringstyle=\color{myorange},
  basicstyle=\ttfamily\small\color{mygray},
  breakatwhitespace=false,
  breaklines=true,
  captionpos=t,
  keepspaces=true,
  numbers=left,
  numbersep=5pt,
  showspaces=false,
  showstringspaces=false,
  showtabs=true,
  tabsize=2,
  frame=single
}


\lstset{
  backgroundcolor=\color{white},
  literate={ą}{{\k{a}}}1 {ć}{{\'c}}1 {ę}{{\k{e}}}1 {ł}{{\l}}1 {ń}{{\'n}}1 {ó}{{\'o}}1 {ś}{{\'s}}1 {ź}{{\'z}}1 {ż}{{\.z}}1
}

\hypersetup{
  colorlinks=true,
  linkcolor=black,
  filecolor=magenta,      
  urlcolor=cyan,
}

\hypersetup{breaklinks=true}
\urlstyle{same}
\renewcommand{\lstlistingname}{Kod źródłowy}
\renewcommand{\lstlistlistingname}{Spis kodów źródłowych}
\renewcommand{\figurename}{Rysunek}
\renewcommand{\listfigurename}{Spis rysunków}

\makeatletter
\renewcommand{\maketitle}{%
  \begin{titlepage}
    \begin{center}
      \vspace*{2cm}
      {\huge \@title \par}
      \vspace{1.5cm}
      {\large \@author \\ nr 1, gr. 2 \\ 325693 \par}
      \vspace{1.5cm}
      {\large Wydział Geodezji i Kartografii\\ Politechnika Warszawska \par}
      \vspace{11cm}
      {\large \@date \par}
      \vspace{1.5cm}
    \end{center}
  \end{titlepage}
}
\makeatother
    \title{\textbf{SPRAWOZDANIE Z ĆWICZENIA 3:}\\ Przeniesienie współrzędnych geodezyjnych na powierzchni elipsoidy obrotowej}
\author{Maja Kret}
\date{Warszawa, \today}


\setlength{\parindent}{0cm}
\setlength{\parskip}{1ex plus 0.5ex minus 0.2ex}
\linespread{1.3}


\begin{document} 
\pagestyle{fancy}
\fancyhf{}
\rfoot{\thepage}
\renewcommand{\headrulewidth}{0pt}
\maketitle
\rhead{~}

\tableofcontents
\newpage

\section{Cel ćwiczenia}
Celem ćwiczenia jest przeniesienie współrzędnych geodezyjnych na powierzchni elipsoidy obrotowej
oraz wizualizacja i analiza ich na mapie.


\section{Wstęp teoretyczny}

\subsection{Algorytm Kivioja}
Algorytm Kivioja umożliwia przeniesienie współrzędnych geodezyjnych, czyli wyznaczenie współrzędnych punktu końcowego oraz azymutu na końcu odcinka linii geodezyjnej. 
Dane do wykonania obliczenia to współrzędne punktu początkowego, azymut do punktu końcowego oraz odległość geodezyjna pomiędzy punktami.
Algorytm wykorzystuje metodę całkowania numerycznego - dzieli linię geodezyjną na n krótkich odcinków, a następnie iteracyjne oblicza przyrosty współrzędnych kolejnych punktów oraz
korekty azymutu aż do punktu końcowego.
Współrzędnymi punktu końcowego jest suma przyrostów współrzędnych.

\subsection{Algorytm Vincentego}
Algorytm Vincentego jest bardziej złożony od algorytmu Kivioja. Rozwiązuje on problem przeciwny, ponieważ za pomocą współrzędnych dwóch punktów 
oblicza azymut oraz odległość pomiędzy nimi. W tym ćwiczeniu użyto go do korekty współrzędnych obliczonych algorytmem Kivioja.

\section{Dane do ćwiczenia}
\vspace*{1cm}

\begin{table}[!ht]
\centering
\begin{tabular}{|c|c|}
\hline
$\varphi_1$ & $53 \degree 45' 00.00000''$ \\ \hline
$\lambda_1$ & $15 \degree 15' 00.00000''$ \\ \hline
\end{tabular}
  \centering
  \caption{Współrzędne punktu początkowego $P_1$
\label{p1}}
\end{table}

\begin{table}[!ht]
  \centering
    \begin{tabular}{|c|c|c|}
    \hline
    nr & długość $s[km]$ & azymut $A[\degree]$\\ \hline
    1 - 2 & 40 & $0\degree 00' 00.000''$ \\ \hline
    2 - 3 & 100 & $90\degree 00' 00.000''$ \\ \hline
    3 - 4 & 40 & $180\degree 00' 00.000''$ \\ \hline
    4 - 1 & 100 & $270\degree 00' 00.000''$ \\ \hline
    \end{tabular}
  \caption{Parametry linii geodezyjnych 
  \label{linie_geo}}
\end{table}

\newpage
\justify
\section{Przebieg ćwiczenia}

\begin{enumerate}
  \item \textbf{Wyznaczenie współrzędnych punktów metodą Kivioja:} Punkty 2, 3, 4 i 1* zostały obliczone
  poprzez zastosowanie funkcji \texttt{kivioj} (kod \ref{kod:kivioja}), która za argumenty przyjmuje współrzędne punktu początkowego,
  azymut do kolejnego punktu, długość linii pomiędzy punktami oraz liczbę iteracji ustawioną na 1000.
  \item \textbf{Analiza punktów na mapie:} Przedstawiono obliczone punkty na mapie i narysowany figurę,
  którą wyznaczają (kod \ref{kod:kivioja}). Następnie obliczono odległość pomiędzy punktami 1 i 1*.
  \item \textbf{Zastosowanie algorytmu Vincentego:} Z użyciem uzyskanych współrzędnych punktu 4 oraz punktu początkowego, 
  obliczono faktyczną odległość pomiędzy punktem 4 a 1 oraz azymut 4 - 1 i odwrotny (funkcja \texttt{vincenty}, kod \ref{kod:vincenty}). 
  Następnie zastosowano te parametry do zamknięcia trapezu - wyliczenia punktu 1* algorytmem Kivioja, który będzie się pokrywał z punktem 1.
  \item \textbf{Wizualizacja na mapie:} Zaznaczono obliczone punkty na mapie oraz narysowano zamkniętą figurę.
   Do stworzenia map oraz tabel użyto biblioteki \texttt{plotly.graph\_objects}
  \item \textbf{Obliczenie pola i obwodu figury:} Do obliczenia pola i obwodu figury użyto funkcji \\
  \texttt{geometry\_area\_perimeter} z biblioteki \texttt{pyproj} (kod \ref{kod:pole}).
\end{enumerate}

\section{Wyniki ćwiczenia}

\subsection{Algorytm Kivioja}
\begin{figure}[H]
  \centering
  \includegraphics[width=1\textwidth]{photos/wspol_kiv.png}
  \caption{Współrzędne punktów obliczone algorytmem Kivioja}
  \label{wspol:kiv}
\end{figure}

W tabeli \ref{wspol:kiv} przedstawiono obliczone współrzędne punktów - szerokość i długość geodezyjną oraz azymut na końcu linii geodezyjnej.


\begin{figure}[!h]
  \centering
  \includegraphics[width=1\textwidth]{photos/scattermapbox.png}
  \caption{Otrzymana figura}
  \label{fig:kiv}
\end{figure}

Na mapie wyraźnie widać, że punkty nie zamykają się w czworokąt, a punkt 1* nie pokrywa się z początkowym.
Punkt 1* znajduje się $1' 9.44748''$ na południe i $46.31182''$ na wschód od punktu 1.
Co za pomocą funkcji \texttt{line\_length} można wyliczyć na odległość aż $2322.516 m$, czyli ponad $2.3 km$.
Rysowane boki trapezu nie są do siebie prostopadłe, z obliczonych azymutów wynika, że linia 2 - 3 była narysowana pod kątem mniejszym niż
$90\degree$, a linia przeciwna 4 - 1* pod kątem większym niż $90\degree$ w stosunku do poprzednich linii.
Natomiast w przypadku linii o długości 40 km, czyli  1* - 2 oraz 3 - 4, azymut przy końcu linii pozostał taki sam i nie 
doszło tu do żadnej korekty.

\subsection{Algorytm Vincentego}

\begin{figure}[H]
  \centering
  \includegraphics[width=1\textwidth]{photos/vincenty.png}
  \caption{Właściwa odległość oraz azymut 4 - 1}
  \label{az:vincenty}
\end{figure}

Przy pomocy algorytmu Vincentego wyliczono odległość oraz azymut pomiędzy punktami 4 i 1, które znajdują się w tabeli \ref{az:vincenty}. Dane te można następnie
użyć do narysowania zamkniętego trapezu o linii 4 - 1 odpowiednio skorygowanej, aby była równoległa do linii 2 - 3.

\begin{figure}[!h]
  \centering
  \includegraphics[width=1\textwidth]{photos/wspol_vin.png}
  \caption{Właściwe współrzędne zamkniętej figury}
  \label{wspol:vincenty}
\end{figure}

Na podstawie współrzędnych zawartych w tabeli \ref{wspol:vincenty} można stwierdzić, że współrzędne punktu pierwszego oraz ostatniego są identyczne,
a więc tworzą one zamknięty trapez. Jego pole oraz obwód obliczono przy użyciu funkcji \texttt{geometry\_area\_perimeter} z biblioteki \texttt{pyproj}.
Pole wynosi \textbf{4 016 880 873.853 $m^2$}, czyli w przybliżeniu $4016.881 km^2$, 
a obwód \textbf{280 862.551$m$} co odpowiada około $280.863 km$. 

\begin{figure}[!h]
  \centering
  \includegraphics[width=1\textwidth]{photos/scattermapbox_vin.png}
  \caption{Otrzymany trapez}
  \label{fig:vincenty}
\end{figure}

\section{Wnioski}

Powodem, dlaczego otrzymana figura po zastosowaniu algorytmu Kivioja się nie zamyka jest kształt Ziemi.
Rysując prostą linię geodezyjną na powierzchni elipsoidy, trzeba brać pod uwagę zakrzywienie Ziemi i dokonywać
ciągłej korekcji azymutu. Jest to zaimplementowane w algorytmie Kivioja, który uwzględniając zakrzywienie,
wylicza współrzędne z skorygowanym azymutem, który różni się od przyjętego na początku.
Odchyłki kształtu figury od oczekiwanego prostokąta wynikają z odległości punktów od biegunów, gdzie zakrzywienia są największe. 
Gdy przyjmiemy za współrzędne punktu początkowego wartości $(0 \degree, 0 \degree)$
różnice współrzędnych pomiędzy punktem pierwszym a ostatnim będą niewielkie. Również azymuty przy końcach linii geodezyjnych
będą bliższe azymutom danym. Pole trapezu będzie bliskie polu prostokąta, ponieważ będzie wynosić 4000.135 $km^2$.
Gdy zmienimy współrzędne punktu początkowego na bliskie równikowi, linie 
2 - 3 oraz 4 - 1* będą rysowane z większym zakrzywieniem, a punkty 1 i 1* będą jeszcze dalej od siebie. 
Wynika z tego, że na dokładność obliczeń bezpośredni wpływ ma szerokość geograficzna punktów.
Natomiast niezależnie od długości geograficznej, linie 1 - 2 oraz 3 - 4 będą zawsze równoległe, a azymut przy ich końcu nie będzie korygowany
dla długości 40km.

\newpage
\section{Kod źródłowy}

\begin{lstlisting}[language=Python, caption=Zamiany jednostek kątowych, label = kod:katy, style = mycode]
  degree_sign = u"\N{DEGREE SIGN}"

  # Radiany na stopnie, minuty i sekundy
  def rad2dms(rad):
    dd = np.rad2deg(rad)
    dd = dd
    deg = int(np.trunc(dd))
    mnt = int(np.trunc((dd-deg) * 60))
    sec = ((dd-deg) * 60 - mnt) * 60
    mnt = abs(mnt)
    sec = abs(sec)
    if sec > 59.99999:
        sec = 0
        mnt += 1
    sec = f"{sec:.5f}"
    dms = (f"{deg}{degree_sign} {mnt}' {sec}''")
    return dms

  # Stopnie dziesiętne na stopnie, minuty i sekundy
  def deg2dms(dd):
    deg = int(np.trunc(dd))
    mnt = int(np.trunc((dd-deg) * 60))
    sec = ((dd-deg) * 60 - mnt) * 60
    mnt = abs(mnt)
    sec = abs(sec)
    sec = f"{sec:.5f}"
    dms = (f"{deg}{degree_sign} {mnt}' {sec}''")
    return dms
\end{lstlisting}

\begin{lstlisting}[language=Python, caption = Algorytm Kivioja, label = kod:kivioja, style = mycode]
  # Parametry elipsoidy
  a = 6378137
  e2 = 0.00669438002290
  g = Geod(ellps='WGS84')
  
  # Współrzędne punktu początkowego
  phi_1_deg = 53 + 45/60
  lam_1_deg = 15 + 15/60
  s = [40000, 100000, 40000, 100000]
  Az_1_deg = [0, 90, 180, 270]
  
  phi_1 = np.deg2rad(phi_1_deg)
  lam_1 = np.deg2rad(lam_1_deg)
  Az_1 = np.deg2rad(Az_1_deg)
  
  # Promienie główne krzywizny
  def M_and_N(phi):
    sin_phi = np.sin(phi)
    M = a * (1 - e2) / (1 - e2 * sin_phi**2)**(3/2)
    N = a / np.sqrt(1 - e2 * sin_phi**2)
    return M, N
  
  # Algorytm Kivioja
  def kivioj(phi_1, lam_1, Az_1, s, n=1000):
    ds = s / n

    phi = phi_1
    lam = lam_1
    Az = Az_1

    for i in range(n):
      M, N = M_and_N(phi)
      dphi = ds * np.cos(Az) / M
      dAz = ds * np.sin(Az) * np.tan(phi) / N

      phi_mid = phi + dphi / 2
      Az_mid = Az + dAz / 2

      M_mid, N_mid = M_and_N(phi_mid)
      dphi_mid = ds * np.cos(Az_mid) / M_mid
      dlam_mid = ds * np.sin(Az_mid) / (N_mid * np.cos(phi_mid))
      dAz_mid = ds * np.sin(Az_mid) * np.tan(phi_mid) / N_mid

      phi += dphi_mid
      lam += dlam_mid
      Az += dAz_mid

    return phi, lam, Az
  
  phis_kiv = [phi_1_deg]
  lambdas_kiv = [lam_1_deg]
  azimuths_kiv = [Az_1_deg[0]]
  
  phis_kiv_dms = [deg2dms(phi_1_deg)]
  lambdas_kiv_dms = [deg2dms(lam_1_deg)]
  azimuths_kiv_dms = [deg2dms(Az_1_deg[0])]
  
  # Obliczenie współrzędnych  punktów
  for i in range(4):
    phi, lam, Az = kivioj(phi_1, lam_1, Az_1[i], s[i])

    phis_kiv.append(np.rad2deg(phi))
    lambdas_kiv.append(np.rad2deg(lam))
    azimuths_kiv.append(np.rad2deg(Az))

    phis_kiv_dms.append(rad2dms(phi))
    lambdas_kiv_dms.append(rad2dms(lam))
    azimuths_kiv_dms.append(rad2dms(Az))

    phi_1 = phi
    lam_1 = lam

  nr = ['1', '2', '3', '4', '1*']

  # Tabela współrzędnych
  fig = go.Figure(go.Table(
    header = dict(values = ['nr', 'phi', 'lambda', 'azymut']), 
    cells = dict(values = [nr, phis_kiv_dms, lambdas_kiv_dms, azimuths_kiv_dms])
  ))
  fig.update_layout(
    title = 'Współrzędne punktów obliczone algorytmem Kivioja', 
    width = 1000, height = 400
  )
  fig.show()

  # Mapa współrzędnych
  fig = go.Figure(
    go.Scattermapbox(
      lat = phis_kiv,
      lon = lambdas_kiv,
      mode = 'markers+lines',
      marker = dict(size = 10, color = 'red'),
      line = dict(width = 2, color = 'red'),
      text = nr,
      hoverinfo='text'))
  fig.update_layout(
    title = 'Scattermapbox - Algorytm Kivioja',
    mapbox_style = "open-street-map",
    width = 2000, 
    height = 1000,
    mapbox = dict(
      center = go.layout.mapbox.Center(
        lat = phi_1_deg,
        lon = lam_1_deg
      ),
      zoom = 7
    ),
  )
  fig.show()

\end{lstlisting}

\newpage
\begin{lstlisting}[language=Python, caption = Algortym Vincentego, label = kod:vincenty, style = mycode]
  # Funkcja udostępniona przez prowadzącego - algorytm Vincentego
  def vincenty(BA,LA,BB,LB):
    b = a * np.sqrt(1-e2)
    f = 1-b/a
    dL = LB - LA
    UA = np.arctan((1-f)*np.tan(BA))
    UB = np.arctan((1-f)*np.tan(BB))
    L = dL
    while True:
      sin_sig = np.sqrt((np.cos(UB)*np.sin(L))**2 +\
        (np.cos(UA)*np.sin(UB) - np.sin(UA)*np.cos(UB)*np.cos(L))**2)
      cos_sig = np.sin(UA)*np.sin(UB) + np.cos(UA) * np.cos(UB) * np.cos(L)
      sig = np.arctan2(sin_sig,cos_sig)
      sin_al = (np.cos(UA)*np.cos(UB)*np.sin(L))/sin_sig
      cos2_al = 1 - sin_al**2
      cos2_sigm = cos_sig - (2 * np.sin(UA) * np.sin(UB))/cos2_al
      C = (f/16) * cos2_al * (4 + f*(4 - 3 * cos2_al))
      Lst = L
      L = dL + (1-C)*f*sin_al*(sig+C*sin_sig*(cos2_sigm+\
        C*cos_sig*(-1 + 2*cos2_sigm**2)))
      if abs(L-Lst)<(0.000001/206265):
          break

    u2 = (a**2 - b**2)/(b**2) * cos2_al
    A = 1 + (u2/16384) * (4096 + u2*(-768 + u2 * (320 - 175 * u2)))
    B = u2/1024 * (256 + u2 * (-128 + u2 * (74 - 47 * u2)))
    d_sig = B*sin_sig * (cos2_sigm + 1/4*B*(cos_sig*(-1+2*cos2_sigm**2)\
      - 1/6 *B*cos2_sigm * (-3 + 4*sin_sig**2)*(-3+4*cos2_sigm**2)))
    sAB = b*A*(sig-d_sig)
    A_AB = np.arctan2((np.cos(UB) * np.sin(L)),(np.cos(UA)*np.sin(UB) - np.sin(UA)*np.cos(UB)*np.cos(L)))
    A_BA = np.arctan2((np.cos(UA) * np.sin(L)),(-np.sin(UA)*np.cos(UB) + np.cos(UA)*np.sin(UB)*np.cos(L))) + np.pi
    return sAB, A_AB, A_BA

  length41, az41, az_odw41 = vincenty(
    np.deg2rad(phis_kiv[3]), np.deg2rad(lambdas_kiv[3]), 
    np.deg2rad(phis_kiv[0]), np.deg2rad(lambdas_kiv[0]))
  length = [f"{length41:.5f}"]
  az_deg = rad2dms(az41)
  az_odw_deg = rad2dms(az_odw41)
  
  # Tabela odległości i azymutów 4 - 1
  fig = go.Figure(go.Table(
    header = dict(values = ['A - B', 'Odl. AB [m]', 'Az. AB', 'Az. odw. AB']), 
    cells = dict(values = ['4 - 1', length, az_deg, az_odw_deg])
  ))
  fig.update_layout(
    title = 'Algorytm Vincentego', 
    width = 1000, height = 300
  )
  fig.show()
  
  phis_vin = phis_kiv[0:4]
  lambdas_vin = lambdas_kiv[0:4]
  azimuths_vin = azimuths_kiv[0:4]
  
  phis_vin_dms = phis_kiv_dms[0:4]
  lambdas_vin_dms = lambdas_kiv_dms[0:4]
  azimuths_vin_dms = azimuths_kiv_dms[0:4]
  
  phi_vin5, lam_vin5, Az_vin5 = kivioj(
    np.deg2rad(phis_kiv[3]), np.deg2rad(lambdas_kiv[3]), az41, length41)
  phis_vin.append(np.rad2deg(phi_vin5))
  lambdas_vin.append(np.rad2deg(lam_vin5))
  azimuths_vin.append(np.rad2deg(Az_vin5))
  
  phis_vin_dms.append(rad2dms(phi_vin5))
  lambdas_vin_dms.append(rad2dms(lam_vin5))
  azimuths_vin_dms.append(rad2dms(Az_vin5))
  
  # Tabela skorygowanych współrzędnych
  fig = go.Figure(go.Table(
    header = dict(values = ['nr', 'phi', 'lambda', 'azymut']), 
    cells = dict(values = [nr, phis_vin_dms, lambdas_vin_dms, azimuths_vin_dms])
  ))
  fig.update_layout(
    title = 'Współrzędne właściwe punktów obliczone algorytmem Vincentego', 
    width = 1000, height = 400
  )
  fig.show()
  
  # Mapa skorygowanych współrzędnych
  fig = go.Figure(
    go.Scattermapbox(
      lat = phis_vin,
      lon = lambdas_vin,
      mode = 'markers+lines',
      marker = dict(size = 10, color = 'blue'),
      line = dict(width = 2, color = 'blue'),
      text = nr,
      hoverinfo='text'))
  fig.update_layout(
    title = 'Scattermapbox - Algorytm Vincentego',
    mapbox_style = "open-street-map",
    width = 2000, 
    height = 1000,
    mapbox = dict(
      center = go.layout.mapbox.Center(
        lat = phi_1_deg,
        lon = lam_1_deg
      ),
      zoom = 7
    ),
  )
  fig.show()

\end{lstlisting}

\begin{lstlisting}[language=Python, caption = Pole i obwód figury, label = kod:pole, style = mycode]
  # Obliczenie pola i obwodu
  area, perimeter = g.geometry_area_perimeter(
    Polygon(
      LineString([
        Point(lambdas_vin[i], phis_vin[i]) for i in range(len(lambdas_vin))])
      ))

  area = abs(area)
  area_km = area / 1000000
  perimeter_km = perimeter / 1000
  area = f"{area:.3f}"
  area_km = f"{area_km:.3f}"
  perimeter = f"{perimeter:.3f}"
  perimeter_km = f"{perimeter_km:.3f}"

  fig = go.Figure(go.Table(
    header = dict(values = [r'','Pole', r'Obwód']), 
    cells = dict(values = [['[km]', '[m]'], [area, area_km], [perimeter, perimeter_km]])))
  fig.update_layout(
    title = 'Pole i obwód wielokąta', 
    width = 1000, height = 300)
  fig.show()
\end{lstlisting}

\newpage
\listoftables
\listoffigures
\lstlistoflistings

\end{document}